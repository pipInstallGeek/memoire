\makeglossaries

%******************************************************************************
%******************************** GLOSSAIRE ***********************************
%******************************************************************************

\newglossaryentry{latex}
{
    name=latex,
    description={Is a markup language specially suited 
    for scientific documents; \lipsum[32]}
}

\newglossaryentry{EPISEN}
{
    name=EPISEN,
    description={{\bf Une école sympa, je pense que vous êtes d'accord avec cela ?}; \lipsum[30]}
}

%******************************************************************************
%******************************** Acronymes ***********************************
%******************************************************************************

\newacronym{gcd}{GCD}{{\bf \textsc{g}}reatest {\bf \textsc{c}}ommon {\bf \textsc{d}}ivisor~; {\bf Rappel~:} de maths pour ceux au fond de salle que blablabla~; Quisque mi lorem, pulvinar eget, egestas quis, luctus at, ante. Proin auctor vehicula purus. Fusce ac nisl aliquam ante hendrerit pellentesque. Class aptent taciti sociosqu ad litora torquent per conubia nostra, per inceptos hymenaeos. Morbi wisi. Etiam arcu mauris, facilisis sed, eleifend non, nonummy ut, pede. Crasut lacus tempor metus mollis placerat. Vivamus eu tortor vel metus interdum malesuada.}

\newacronym{llm}{LLM}{{\bf \textsc{l}}arge {\bf \textsc{l}}anguage {\bf \textsc{m}}odel~; {\bf Rappel~:} pour ceux qui ne lisent aucun journal que blablabla~; Duis nisl nibh, laoreet suscipit, convallis ut, rutrum id, enim. Phasellus odio. Nulla nulla elit, molestie non, scelerisque at, vestibulum eu, nulla. Ut odio nisl, facilisis id, mollis et, scelerisque nec, enim. Aenean sem leo, pellentesque sit amet, scelerisque sit amet, vehicula pellentesque, sapien.}
